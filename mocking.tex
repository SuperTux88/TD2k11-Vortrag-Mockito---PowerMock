\documentclass{beamer}

% uncomment to generate printer friendly scaled version
%\usepackage{pgfpages}
%\pgfpagesuselayout{4 on 1}[a4paper,border shrink=5mm,landscape]
\usepackage[utf8]{inputenc}
\usepackage[ngerman]{babel}
\usepackage[T1]{fontenc}
\usepackage{graphicx}

%\usetheme{}
%%%%%THEME_PLACEHOLDER_JENKINSBUILD%%%%%

\setbeamercovered{transparent}
%\setbeamertemplate{footline}[frame number]

\title{Test driven development mit JUnit, Mockito und PowerMock}
\institute{Computerseminar Tondorf 2011}

\author[F. Becker, B. Neff]{
        Felix Becker \& 
	Benjamin Neff
}

\begin{document}
	\begin{frame}
		\titlepage
	\end{frame}

	\begin{frame}
		\frametitle{Agenda}
		\setcounter{tocdepth}{1}
		\tableofcontents
	\end{frame}
	
	%
	% Einfuehrung
	%

	\section{Einführung}
	
		\begin{frame}
			\frametitle{Was ist Unit-Testing?}
		\end{frame}

		\begin{frame}
			\frametitle{Was ist Test driven development?}
		\end{frame}

	%
	% Tools und Frameworks
	%
	
	\section{Tools \& Frameworks}
		\begin{frame}
			\frametitle{Frameworks}
		\end{frame}

		\begin{frame}
			\frametitle{Continuous Integration}
		\end{frame}

		\begin{frame}
			\frametitle{ECL emma}
		\end{frame}

	%
	% Probleme 
	%
	
	\section{Probleme}
		\begin{frame}
			\frametitle{Refactoring}
		\end{frame}

		\begin{frame}
			\frametitle{Nicht testbares Verhalten}
		\end{frame}

		\begin{frame}
			\frametitle{Beispiele}
		\end{frame}

	%
	% Mockito und PowerMock
	%
	
	\section{Mockito \& PowerMock}


		\subsection{Problem 1}

			\begin{frame}
				\frametitle{Problem}
			\end{frame}

			\begin{frame}
				\frametitle{Loesung}
			\end{frame}


		\subsection{Problem 2}

			\begin{frame}
				\frametitle{Problem}
			\end{frame}

			\begin{frame}
				\frametitle{Loesung}
			\end{frame}

		\subsection{Common Pitfalls}

			\begin{frame}
				\frametitle{Falsches Benutzen}
				%.whenNew(Socket.class).with...
			\end{frame}

\end{document}
